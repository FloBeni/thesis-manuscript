
\chapter*{Preamble}

% \vspace{-60pt}
\vspace{-5pt}

Four billion years of evolution have shaped animal genetic materials, \textit{i.e.} genomes, and human seeks to uncover what is biologically functional in them. Notably, metazoan genomes are complex and vary in size, base composition, expression, number of genes, structure \textit{etc}. I will open the introduction in \hyperref[part:intro]{Part I} with \hyperref[chap:intro-metazoan_genome]{Chapter 1} by illustrating the complexity and diversity of metazoan genomes architecture and expression. In order to discern what is functional in these genomes, it is imperative to understand the evolutionary forces that have shaped them over time.
\vspace{-19pt}
\begin{center}
\mbox{\textit{`Nothing in Biology Makes Sense Except in the Light of Evolution' T.Dobzhansky (\textcolor{BLUEROYAL}{1973})}}
\end{center}

I present in \hyperref[chap:intro-evol_forces]{Chapter 2} my scientific area, evolutionary biology, where we explore genomes and the main evolutionary forces that affect them: mutation, selection and drift. Notably, since Darwin in 1859, who elucidated the role of natural selection in species evolution, successive theories highlighted that natural selection cannot be responsible for all changes. Indeed, neutral or slightly neutral changes affect genomes, and drift hazard may disturb the efficiency of selection to promote slightly advantageous mutations or purge deleterious ones. However, because of the innate desire to discover meaningful characteristics of genomes, we tend to jump to conclusions and prematurely attribute changes solely to natural selection. This, eventually, created vigorous debates within the scientific community between neutralist and selectionist views of evolution. 

Fortunately, we live in a period of surrounding data, with an unprecedented amount of sequenced genetic materials available for study, that I present in \hyperref[chap:intro-biol_evol]{Chapter 3}. This wealth of data, combined with advances in methodology and technology, allowed the biologist to re-investigate evolutionary theories.

In this powerful context, I present in \hyperref[chap:intro-objectives]{Chapter 4} how I intend to bring new answers to the neutralist/selectionist debate. Indeed, the “drift barrier" hypothesis, which predicts that reduced drift leads to more optimized genomes, offers me a great conceptual framework for discerning adaptive and non-adaptive traits by studying the impact of random genetic drift on genomes architecture.

The \hyperref[part:studies]{Part II} outlines the methodology and findings of my thesis in three articles. Initially, I gathered genomic and transcriptomic data, which underwent integrative bioinformatic analyses to facilitate comparative studies. Thus, I introduce GTDrift in \hyperref[chap:chap5-GTDrift]{Chapter 5}, a comprehensive data resource housing transcriptomes analyses alongside effective population size proxies (\Ne). This resource serves for exploring the impact of drift on genome architecture. In \hyperref[chap:AlternativeSplicing]{Chapter 6}, I study how transcriptome diversity is affected by drift, through the quantification of alternative splicing in metazoans. \hyperref[chap:CodonUsage]{Chapter 7} focuses on elucidating the reasons behind the paucity of translational selection in metazoans.

Finally, I discuss in \hyperref[chap:conclusion]{Part III} the implications of this thesis, emphasizing the necessity for reproducibility in a world surrounded by data and analyses, to maintain scientific integrity and ensure qualitative research.

% All figures, scripts and \LaTeX source code used in this manuscript can be reused under CC-BY-SA license, available at \url{https://github.com/FloBeni/thesis-manuscript}.