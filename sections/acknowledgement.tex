\thispagestyle{empty}
\vspace*{-80pt}

\section*{\large{Remerciements}}
\vspace*{-5pt}

Je tiens à remercier toutes les personnes qui ont fait de ces années un temps d'
épanouis-sement personnel et scientifique.

\vspace*{7pt}
\textbf{Aux amis,} 
toujours là, m'inspirant par leurs qualités, avec qui les heures en dehors du labo sont si ressourçantes et nécessaires pour me libérer l'esprit.

À Bobo/Gégé pour ces 28 ans de délires, les missions en Irlande, Pologne, stratedgic object, Italie, les tennis, la spéléo, mais pas trop... Ton goût pour l'aventure me pousse à oser sortir de ma zone de confort. Je garde l'espoir d'une promotion.

Le Doud, depuis le collège t'as toujours été là, merci pour ces journées tt/AOE, pour m'avoir introduit à la team CPE. Ton calme, ton organisation, ton professionnalisme et ton game play à fifa m'ont depuis longtemps inspirés. Merci aussi de me backuper avec ton aide en informatique, en ML, et pour les travaux... On le fera ce micro-ondes.

Lucas, poto depuis mes débuts au LBBE il y a 6 ans, tu es pour beaucoup responsable de mon rapport à la vie, à la relativisation, et au lâcher-prise. Merci pour tous ces moments, pétanque, Sucre, fifa, techno, festoch \textit{etc.} À toi et à Kazou pour m'avoir fourni des nuits d'hôtel accompagnées de grandes discussions savantes pendant le Covid !

Vic, pour tous ces superbes moments qu'on a partagés à Nice. Franchement, lors de mes débuts au laboratoire, passer de tels weeks-ends dans le sud était vraiment top. Tes projets pro à l'étranger ont indéniablement nourri mon envie de partir travailler au Canada.

La team Paris évidemment, Seb, Vince et Flo ; les bivouacs, les ampoules, les lynxs, BK production \textit{etc.} C'est toujours un plaisir de passer des moments avec vous, rires et activités garantis.

Au come-back pour les belles vacances passées chaque été et Nouvel An dans des lieux toujours plus stylés.

Flo la rencontre techno, merci à toi, Val et Florent, à ces dimanches après-midi sur la terrasse du Sucre avant d'attaquer la semaine, tes divines pâtisseries, ta bonne humeur et tes pitreries.


\vspace*{7pt}
\textbf{Aux collègues devenus amis}, Thibault, pour m'avoir intégré dans le domaine de la bioinfo avec tant de bienveillance, me délivrant tant de tips. À cette fameuse baignade dans la grotte lors de la 1$^\text{ère}$ inté avec Judith, Alex et Théo. À nos pique-niques/pétanque sur les quais au coucher de soleil.

Merci à la team des docs/postdoc du 2$^\text{ème}$ à mon arrivée au labo, Alexia, Alex, Dji, Théo, Diego, Marina. Vraiment de bons moments passés. Rien que cette première conférence Lille into Bruxelles était exceptionnelle (surface plane). Je me suis rapidement aperçu que les années à venir allaient être formidables avec vous dans cet environnement de travail.

Gasp, l'élève à dépasssé le maître sur RL, tu peux remercier le Covid. C'est regrettable que tu en sois venu à le renier et gâcher tant de talent. Merci pour ton aide à l'animation au labo, et pour ces games avec Abi et Thibaud, que je remercie également.

Aux potes plus récents, à la team Mroc, aux soirées jeux, aux crémaillères désordonnées \textit{etc.}: Mélo, Matthieu, Soso, Lulu, Jess, Emma, Adrian, PA, Luchad le pgm EVA, Amandine, Rémi, Léa, Max, Valoche.


\vspace{17pt}\textbf{À ma famille,}
Mams, Pat, merci de m'avoir donné l'opportunité de faire tant d'études, dans un cadre si privilégié, de m'avoir toujours poussé vers l'avant et donné les moyens d'y arriver. Ton carriérisme, Mams, m'apprend à persévérer pour suivre mes passions. Pat, ton sens extrême de l'organisation coule dans mes veines et me permet de rester ordonné face aux grandes quantités de données que j'ai à gérer.

Quentine/bb mousse pour m'avoir initié aux basses, et Pau, pour ces étés au soleil et nos moments en famille.

Papi, Mamie, c'est bien dans votre campagne que j'ai commencé à m'intéresser à la biologie, en passant tant de temps dans la nature, à courir après les vaches et à pêcher la truite... Merci.

Didou, merci pour ton soutien et les jours passés à réviser dans l'Antre des verts. Ce fut un plaisir d'avoir été salarié d'ELCO.


\vspace*{7pt}
\textbf{À la team LEHNA,}
malgré le peu de temps passé dans vos locaux et des sujets assez différents du reste de l'équipe, je garde de très bons souvenirs. Vous avez été très accueillants, et j'ai pu avoir des échanges enrichissants et bienveillants. Merci pour la journée d'équipe, la sortie canoë \textit{etc.} Spécial merci à Clémentine, Maïlys, Axelle, Héloïse, Gautier, Nans, Paul, et Emma.


\vspace*{7pt}
\textbf{Au LBBE,}
je ne compte plus les personnes qui m'ont aidées, soutenues, avec qui j'ai eu des discussions, qui m'ont permis d'évoluer, et de faire évoluer ma façon de faire de la recherche.
Les pôles du LBBE, administratif évidemment: Nathalie merci de gérer tout cela à la perfection malgré la multitude de contrats que j'ai eu dans ce labo. Merci pour ce travail qui nous déleste de ces tracas administratifs.
Bien sûr, merci au pôle informatique, Stéphane et Bruno, je vous ai suffisamment dérangés avec le cluster, iRods, le pbil, les VMs, Shiny \textit{etc.} Si j'ai pu réaliser autant d'analyses c'est parceque l'infrastructure à laquelle nous avons accès est vraiment entre de bonnes mains. Au fil des années j'ai pu apprécier les efforts d'amélioration qui ont été proposés, et c'est un bonheur de savoir que l'on peut compter sur vous et venir vous déranger n'importe quand pour la moindre question, et d'en sortir avec plus de solutions qu'on avait de questions. Merci évidemment aussi à Philippe de prendre autant de temps pour m'expliquer les choses avec bienveillance et pédagogie.

Merci à la team HH/LBBistrot, à Théo pour nous avoir passé la main, et donc à Gasp, Nat et Pascal de m'avoir rejoint. Si c'est tant une réussite c'est bien grâce à vous, et c'était un plaisir de gérer ça avec vous.

Aux Pinsons, merci de proposer autant d'activités renforçant la cohésion au labo.

Merci à nos DU Fabrice et Manu pour faire de ce labo un espace où il fait bons vivre et laisser/encourager les prises d'initiatives.

\textbf{La team BPGE}, je garde de superbes souvenirs, à mon arrivée en 2018, de cette retraite au lac de Cublize, qui m'a directement plongé dans une équipe top, pleine de bienveillance et de projets. Avec tant d'experts dans différentes disciplines de la biologie évolutive, ce fut un cadre de travail dynamisé par de nombreuses réunions, scientifiquement enrichissantes, avec une recherche constante de rigueur.

Merci à ceux à qui j'ai pu demander de l'aide et échanger au cours de ces années : Alba Marino, Olivier Arnaiz, Olivier Tenaillon, Maud Tenaillon, Ignacio Bravo, Julien Clavel, Laurent Gueguen, Guillaume Achaz, Marie Sémon \textit{etc.}

Merci à ceux qui m'ont ouvert les portes de la recherche et de la science : Elie Maza, Roland Barriot, Laurent Bezin, Simon Hardy, et Patrick Desrosiers.

Merci aux membres de mon comité de pilotage, Benoit, Emiliano et Vincent, pour vos retours constructifs et les discussions sur les perspectives de carrière encourageantes. Merci Rita, tu as été une super tutrice, disponible, à l'écoute, motivante, bienveillante et de bon conseil.

Aux rapporteurs, Stéphanie et Hugues, merci d'avoir accepté de relire ce manuscrit et pris le temps de le corriger. Anna-Sophie et Sabine, merci d'avoir accepté d'être membre de mon jury.

\vspace*{7pt}
\textbf{Tristan}, merci de m'avoir donné l'oppportunité de réaliser cette thèse, merci pour nos échanges aussi bien scientifiques qu'humains, et surtout merci pour les relectures de ce manuscrit.

\vspace*{7pt}
\textbf{Anouk}, que te dire ? Depuis le début de cette aventure, tu es très, voire trop bienveillante avec moi. Merci d'avoir pris le temps de me former techniquement, de m'avoir transmis l'importance d'être organisé. Merci pour nos collaborations, pour toutes les discussions, les quelques matchs de tennis, les paniers légumes \textit{etc.} Merci d'avoir été à l'écoute, une confidente à qui je pouvais parler si ça n'allait pas. Tu es grandement responsable de mon bien-être au labo, et de mon épanouissement scientifique. Finalement, je te considère comme une encadrante de cette thèse tant tu y as contribué.


\vspace*{7pt}
\textbf{Laurent}, alias chef, évidemment que je te dois vraiment beaucoup voir tout. Merci de m'avoir accordé autant de confiance et de m'avoir gardé à tes côtés pendant toutes ces années.
J'ai eu la chance d'avoir un superviseur toujours disponible, malgré les responsabilités que tu as, accessible d'une simple rotation de chaise pour discuter que ce soit science, carrière ou perso. Notamment, j'ai souvenir pendant le Covid où je n'avais qu'à envoyer un `\textit{dans 5min sur discord ?}' par mail, et nous voilà en visio pour discuter science ou société, environ une fois par jour voire plus. Tu as fait de cette période une phase épanouissante scientifiquement.
Tu es extrêmement patient et pédagogue, m'expliquant les choses sans jamais paraître agacé et en prenant le temps. Tu m'as toujours fasciné par la richesse de ton savoir, et surtout par ton humilité, ce qui m'a nécessairement fait grandir scientifiquement et humainement. 
J'étais constamment en présence d'une montagne de savoir pour laquelle je me plaisais à être l'expert technique et à contribuer à ses interrogations. Nous avons certes pu avoir des désaccords sur la facon de valoriser nos résultats dans un domaine toujours plus compétitif, mais tu m'as fourni un cadre de travail incroyable, que j'espère retrouver ailleurs. Ce fut un vrai plaisir de travailler pour et avec toi pendant ces 6 années. 
Enfin, merci pour le nombre de relectures, que ce soit pour ce manuscrit ou bien nos articles scientifiques. Merci pour tous tes retours. Un jour, peut-être mais j'en doute, j'atteindrai ce niveau d'excellence. 

\newpage
% \vspace*{7pt}
Et enfin, merci à \textbf{mon Beb}. Merci \textbf{Mel} d'être toujours là, radieuse, solaire, altruiste, et tellement joviale. Si je peux avoir des hauts et des bas, que ce soit lié au boulot ou autre, tu es, toi, toujours souriante et à l'écoute. Tu es en quelque sorte ma psy personnelle, et une scientifique/collaboratrice à l'écoute de mes projets. Finalement, nous avons acheté et vécu ensemble dès le début de ma thèse, et ce fut l'une de mes meilleures expériences. C'est si ressourçant de vivre avec toi, notamment pendant cette thèse où moralement, c'était une période moins facile que lors de mes précédents contrats. Tu es toujours partante pour faire des activités, toujours enthousiaste, même lorsqu'il s'agit de partir sur un autre continent ensemble. Je partage tellement de bons moments avec toi, et j'espère en créer tant d'autres. Merci, Je t'aime.

\vspace*{5pt}
Merci à tous ceux que j'aurais pu oublier mais qui, au cours de ces années, ont contribués, directement ou indirectement, à ces travaux de recherche.

\vspace*{10pt}
Depuis ces lignes, je laisse la  place à la science.


\begin{flushright}
    \textit{Florian Bénitière}
\end{flushright}