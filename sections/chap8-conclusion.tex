\thispagestyle{empty}
\chapter[Discussion \& Perspectives]{Discussion \& Perspectives}
{\hypersetup{linkcolor=GREYDARK}\minitoc}

\section{Summary of main results}

During my thesis I analyzed transcriptomic and genomic data, organized in a data resource including almost 16,000 RNA-seq samples and 1,506 species along with proxies of the random  \gls{genetic drift} intensity. These information have been used to study how random  \gls{genetic drift} affects alternative splicing and \gls{translational selection} across metazoans. I summarize in the following sections the main findings of my thesis.


\subsection{Alternative splicing, a genetic burden limited by drift}

In the first scientific study we investigated the alternative splicing products, alternative variants, and their functional significance across several metazoans. We have developed protocols to tackle the question from different angles. The main one was to use the “drift barrier” hypothesis, according to which biological processes within a genome will be optimized up to the limit imposed by  \gls{genetic drift}. Indeed, to rephrase it briefly, Lynch postulated that the genomes of species with small \gls{effective population size} would be subject to more intense  \gls{genetic drift} compared to species with high \gls{effective population size}, thus reducing the effectiveness of selection to purge slightly deleterious mutations in small \acrshort{Ne}~species.

Through the estimation of the \textit{per} intron alternative splicing rates across 53 species, our results demonstrate a negative correlation between alternative splicing and \gls{effective population size}. This relationship is robust to phylogenetic inertia and the quantity of transcriptomic data analyzed. Thus, the increase in the rate of alternative splicing between species (from 0.8\% to 3.8\%) mainly reflects the increase in the intensity of  \gls{genetic drift}, and corresponds to transcription errors whose quantity is modulated by drift.

In a second protocol we identified two categories of introns, rare splice variants (\acrshort{SV}s) representing the vast majority of the repertoire of splicing isoforms (from 62.4\% to 96.9\%) and abundant \acrshort{SV}s.
We observed that abundant \acrshort{SV}s have a strong signal of functionality, indeed up to 70\% are frame preserving compared to 33\% in rare variants, a rate expected if the splice site is randomly selected on the gene. Also, the \acrshort{AS} rate measured on rare \acrshort{SV}s is strongly related to \acrshort{Ne}, as expected under the “drift barrier” hypothesis, which states that errors should increase with decreasing \acrshort{Ne}. This relationship does not hold for the \acrshort{AS} rate measured on abundant \acrshort{SV}s, which are supposed to contain a large proportion of functional transcripts.

Another line of research consisted of studying splice sites constraints, by comparing those of the spliced variants and those of the main isoforms in \textit{Drosophila melanogaster} and \textit{Homo sapiens}. Our results show that, in \textit{Homo sapiens}, the splice sites of the main isoform are constrained, but the spliced variants do not present any particular constraint compared to the control regions. Whereas in \textit{Drosophila melanogaster}, there is selection on splice sites of the most abundant \acrshort{SV}s. These observations also support the hypothesis that \acrshort{AS} products are predominantly non-functional and therefore not under selective constraints, except for abundant \acrshort{SV}s in some species, such as \textit{Drosophila melanogaster}.

Finally, we investigated whether low-expressed genes have more rare variants than high-expressed genes, as we expect them to be purged more efficiently into the latter category if they arise primarily from splicing errors. For most species, the rate of rare variants decreases with gene expression accordingly to our predictions.

All in all, our first study reveals that \acrshort{AS} mainly reflects erroneous transcripts which rate is controlled by the intensity of random  \gls{genetic drift} in metazoans.


\subsection[Translational selection is rare in metazoans: variations in drift and fitness]{Translational selection is rare in metazoans:\\variations in drift and fitness}

In our second scientific study we analyzed \gls{codon usage} variations among metazoans, focusing our analyses on \gls{translational selection} which promotes codons optimizing translation process. Our first observation was that inter-species variations in \gls{codon usage} are strongly influenced processes impacting both coding and non-coding sequences, called neutral \gls{substitution} patterns (\acrshort{NSP}). 

Subsequently, in each species, we identified the set of \gls{codon}s decoded by the most abundant \acrshort{tRNA}s, that we called putative-optimal codons (\acrshort{POC}s), predicted to be codons promoted by \acrshort{TS}. Interestingly, highly expressed genes are enriched in \acrshort{POC}s compared to low expressed genes in most studied species. This enrichment reaches +26\% in \textit{Caenorhabditis elegans}, +14\% in flies, and a mere +3\% in vertebrates. We further showed that constrained sites of a gene tend to overuse \acrshort{POC}s compared to less constrained sites. Additionally, analyses on substitution patterns and polymorphism in \textit{Drosophila melanogaster} reveal that non-\acrshort{POC}s towards \acrshort{POC}s \gls{substitution}s are favored in highly expressed genes compared to lowly expressed genes. These analyses strongly suggest a selection to promote the use of codons that match the \acrshort{tRNA} pool. 

Then, we investigated, for species for which \acrshort{TS} is effective, how the \acrshort{tRNA} pool responds to variations in neutral substitution patterns (\acrshort{NSP}). This question is particularly interesting in Diptera and Lepidoptera because of the strong \acrshort{TS} signal that coexists with large variations in \acrshort{NSP} across species. We demonstrated that the translation machinery is co-adapting to the \acrshort{NSP} changes by modulating both the \acrshort{tRNA} abundance and the \acrshort{tRNA} affinity for a particular \gls{codon}.

Overall, our results show that \acrshort{TS} is scarce in metazoans, with a small population-scaled selection coefficient (\textit{i.e.} $\acrshort{S} < 1$), and that species where \acrshort{NSP} is detected correspond to species with large \acrshort{Ne}. In this range of \acrshort{S} values, the “drift barrier” suggests that \acrshort{Ne}~must be large for selection to be efficient and promote codons optimizing translation. Indeed, in small \acrshort{Ne}~population \acrshort{TS} is barely observed in our dataset. However, while \acrshort{TS} is observed only in large \acrshort{Ne}~species, some large \acrshort{Ne}~species also show no \acrshort{TS} signal.

Finally, we investigated how \acrshort{TS} can become ineffective due to heterogeneous neutral substitution patterns. It appears that species with heterogeneous \acrshort{NSP} do not present \acrshort{NSP} signal, and that \acrshort{TS} is only observed in species with homogeneous \acrshort{NSP}. However, some species with a large \acrshort{Ne}~and a homogeneous \acrshort{NSP} do not exhibit \acrshort{TS} signal, such as some hymenopterans. These results lead us to hypothesize that the selective advantage in optimizing the translation machinery is not the same for all species.


\section{Discussion}

I explore the possible consequences of this thesis on other scientific questions by first presenting how the “drift barrier” hypothesis can be useful for deciphering what is adaptive or not. Then, by presenting how our results could be interesting in applied scientific subjects.

Also, in the following sections, I discuss how we/I, as scientists, can work to bring compelling reproducible data to the community.
I will delve into the accessibility and reproducibility of the data in research with the tools available to the bioinformaticians today, on which I devoted a lot of time to provide all the information necessary for the reproduction of our articles, data and results.


\subsection{The “drift barrier”, an attractive framework}

In biology it is common to study biological processes as if they were adaptive. But we know that the non-adaptive forces cannot be systematically ruled out, and need careful consideration~\citep{lynch_frailty_2007}. In population genetics, the “drift barrier” hypothesis is one of the most attractive concept to examine non-adaptive vs adaptive model. Theoretically slightly deleterious/advantageous mutations with $|$\acrshort{Ne}~$\acrshort{s}| \ll 1 $ propagate in the population as if they are neutral. Thus, if \acrshort{s} is constant a decrease in \acrshort{Ne}~implies that more and more slightly deleterious mutations behave neutral and thus have a greater chance to reach fixation in a species. With the same reasoning, advantageous mutations will behave as neutral and will have less chance of reaching fixation than in large-\acrshort{Ne}~population.

This observation led Lynch to propose that biological processes, as they approach optimality, will encounter a barrier beyond which any further optimization will be hampered by drift~\citep{lynch_genetic_2016}. Indeed, for a trait close to optimality, new beneficial mutations are supposed to have diminished fitness advantages, decreasing \acrshort{s}, and will behave as neutral.

The question is whether this could be observed in nature: are there features of the genome that actually accumulate a slightly deleterious burden, or purge that burden, due to the change in \acrshort{Ne}? Does \acrshort{Ne}~alone determine the level of optimization of biological processes?

First, Lynch observed that the mutation rate per generation was linked to the \acrshort{Ne}~\citep{lynch_evolution_2010, sung_drift-barrier_2012, bergeron_evolution_2023}. Thus, he concluded that selection operates to minimize the mutation rate, with an efficiency limited by random  \gls{genetic drift}. The genome size of Asellid isopods has also been shown to increase as long-term \acrshort{Ne}~decrease, due to an accumulation of repeated elements~\citep{lefebure_less_2017}. However, in some other metazoan clades the predictions are not observed~\citep{whitney_did_2010, roddy_mammals_2021, marino_effective_2024}.

During this thesis we showed that \acrshort{AS} is correlated with  \gls{genetic drift} intensity, supporting the idea that selection tends to optimize a low rate of AS, but that drift keeps it quite high for species with small \acrshort{Ne}. These observations, combined with others, led us to conclude that \acrshort{AS} products are primarily errors in low-\acrshort{Ne}~species. However, in \hyperref[chap:CodonUsage]{Chapter 7}, we showed that in metazoans, \acrshort{Ne}~might be responsible for variations of \acrshort{TS} intensity but is not the only factor. These results suggest that if our measure of \acrshort{Ne}~is correct, and genomes have reached equilibrium, other parameters than \acrshort{Ne}~are at play on translational selection. For instance, the fitness landscape of optimizing translational machinery may differ across species, \textit{i.e.} fast growing species~\citep{manthey_rapid_2024} could have better interest to optimize translation than species with slow growth rate~\citep{rocha_codon_2004}.

Overall, by acknowledging that both selection (\acrshort{s}) and drift (\acrshort{Ne}) are at play, the “drift barrier” provides an interesting framework to ask whether biological processes, or genomic traits, are actually adaptive. For small population-scaled selection coefficient, our studies show that there are cases where the “drift barrier” hypothesis makes it possible to explain why genomic characteristics vary, and how. As such, non-adaptive evolution of certain aspects of genomes architecture cannot be overlooked. With this in mind, human species must be studied with extreme caution, especially since biologists tend to draw sweeping conclusions about the extreme complexity of our genome, when in reality, we are part of the species which exhibit the greatest random  \gls{genetic drift}, making us more vulnerable to the accumulation of genetic burden.


\subsection{The limit of the “drift barrier” approach}

While we showed that drift impacts some fundamental processes that are not under strong selection (\textit{i.e.} small population-scaled coefficient), it is not clear if this test would be appropriate for other traits under stronger selection. If there is a causal relationship between \acrshort{Ne}~and a trait, it seems relevant to interpret what is the adaptive significance of a trait (increasing or decreasing) providing an indication of its biological functionality, which could ideally be complemented by other indicators.

However, if there is no relationship, interpretation is very difficult and requires extreme caution. Indeed, we can invoke different reasons to explain this absence of relationship: the trait is not at equilibrium selection/drift; the drift proxy is noisy; the \acrshort{Ne}~used is not relevant for this trait selection/drift balance (\textit{i.e.} short-term vs long-term \acrshort{Ne}); the fitness landscape varies (\textit{i.e.} not the same interest to optimize a trait in each species). Also, the non-existence of the relationship between drift and a trait variations may simply be a true observation due to the fact that \acrshort{Ne}~varies in a range that does not apply to the “drift barrier” either because this trait is subject to strong selection (\textit{i.e.} $\acrshort{S} \gg 1$), or because this trait evolve neutrally (\textit{i.e.} $\acrshort{S} \ll 0.01$).

We must be careful not to reproduce the same cognitive biases that we criticized previously. This means that we should not over-interpret our results, nor indirectly force expected correlations, but keep in mind that inconclusive results are still results. It is encouraging to observe in the literature that we accept that the hypothesis may not work. For example the most notable variation in genomes architecture is genomes size, and this has recently been shown to not support the “drift barrier” hypothesis~\citep{roddy_mammals_2021, marino_effective_2024}.

Unfortunately, to test this attractive “drift barrier” hypothesis, we only have the combination of \acrshort{Ne}~and \acrshort{s} that biology on Earth offers us. We are in a laboratory where the possibilities for variations of \acrshort{Ne}~and \acrshort{s} are limited, and where many other parameters, that we cannot control, change.


\subsection{Potential consequences for future research}

As mentioned in the introduction, many scientists consider that the primary purpose of alternative splicing is to increase the functional repertoire of genomes, particularly ours~\citep{graveley_alternative_2001, black_mechanisms_2003, pan_deep_2008, nilsen_expansion_2010, blencowe_relationship_2017}. These far-reaching conclusions have already permeated the scientific community without clear evidence, as it can be read in many recent papers that \acrshort{AS} \textit{`contributes to the majority of protein diversity'}~\citep{jiang_alternative_2020, verta_role_2022, singh_importance_2022, manuel_re-evaluating_2023}, with some still pointing out that there is controversy over this~\citep{pozo_assessing_2021, wright_alternative_2022, singh_importance_2022, manuel_re-evaluating_2023}.

These sloppy conclusions have reached pharmaceutical studies, particularly in oncology where \acrshort{AS} is widely studied for its implication in tumor development~\citep{venables_aberrant_2004, kalnina_alterations_2005, srebrow_connection_2006, david_alternative_2010, huang_rna_2020, qi_significance_2020, sciarrillo_role_2020}. In some studies all variants are considered as functional, and disruption in \acrshort{AS} events is interpreted as a loss of function~\citep{schmitz_widespread_2020, cummings_transcript_2020}. But it might be important to keep in mind that most \acrshort{AS} events are actually irrelevant, functionally speaking, and taking this into account could help improve protocols and avoid misinterpreting its results. Our work~\citep{benitiere_random_2024} complements studies that investigate the relative proportion of functional product of AS, concluding that in human most \acrshort{AS} variants are errors and the “one gene many proteins" hypothesis corresponds to rare cases~\citep{pickrell_noisy_2010, gonzalez-porta_transcriptome_2013, tress_most_2017, tress_alternative_2017, saudemont_fitness_2017}. Also, we identified a set of variants that seems to be functionally relevant in most species, \textit{i.e.} the abundant spliced variants. These results appear useful for prioritizing further investigations in more applied research, aimed at studying how \acrshort{AS} modulates \gls{phenotype}s~\citep{verta_role_2022, singh_importance_2022}, diseases~\citep{scotti_rna_2016}, drug development~\citep{ren_alternative_2021} \textit{etc.}

Our second analysis aligns with papers identifying no, or negligible, translational signals in humans or other vertebrates~\citep{mouchiroud_compositional_1988, kanaya_codon_2001, duret_evolution_2002, pouyet_recombination_2017}. Indeed, we searched for \gls{translational selection} in 250 metazoans and found it to be negligible in vertebrate species, contrary to the findings of other papers~\citep{chamary_hearing_2006, gingold_dual_2014, dhindsa_natural_2020}. These articles often lack negative control or they are misinterpreted. For example~\citet{gingold_dual_2014} observed that gene sets belonging to different functional categories have a different \gls{codon usage}, which they interpreted as selection on the translation program for cell proliferation and differentiation. But in fact~\citet{pouyet_recombination_2017} showed that these differences are linked to recombination, a process impacting both coding and non-coding regions, thus unrelated to the translation process. This underscores the necessity to have neutral control when searching for adaptive traits, especially in this controversial case. Interestingly we showed that we can predict a set of \gls{codon}s optimized for translation based on the \acrshort{tRNA} pool. Thereby, just because \gls{codon usage} in primates is not optimized does not mean it cannot be optimized. This latter statement is particularly interesting for a biological field of genome recoding. Indeed, many scientists are working on the incorporation of \gls{synonymous} mutations to improve cellular properties~\citep{singh_genome_2021}, or therapeutic strategies to prevent viral diseases~\citep{martinez_synonymous_2019}. A striking example is the development of the \acrshort{mRNA} vaccine encoding the SARS-CoV-2 Spike during the pandemic. The synthesis of this \acrshort{mRNA} requires choosing which \gls{synonymous} \gls{codon}s to use in order to optimize immunogenicity~\citep{gimenez-roig_codon_2021, lai_viral_2023, zhang_algorithm_2023}. Our work, sheds light on how recoding can be prioritized, by preferentially targeting putative optimal codons, decoded by the most abundant \acrshort{tRNA}, and by taking into account wobble pairing. Our claims are mainly based on dipterans where \gls{codon usage} appears to be much more driven by translational selection than in our genome. However, other species exhibit a non-null population-scaled selection coefficient, meaning that we might indeed, with this protocol, capture codons optimizing translation.

\subsection{Data accessibility}

To convince, researchers need to share as much data used in their paper as they can. Even if the methods and the results presented in a scientific paper are peer reviewed, it is not rare to realise that data are not always shared, or can be erroneous in comparison to what is published. This may be due to the human cost of revising a paper which does not allow a researcher for more time to check this kind of details systematically, but also because many papers do not prioritize data sharing \citep{dance_stop_2023}. In my little experience, I have often encountered this kind of limitation when I wanted to collect data from a paper, which undoubtedly led one to be skeptical of the article itself, as it couldn't even get its hands on the most basic data, like that used in the charts. In this regard, this discredits the message given in a scientific article. An article could share all data, in order to be independent of the author, who may no longer work in the field. Due to technical advances in machine learning and computing, it might be reasonable to expect that in the near future scientific journals will come up with an automated method/pipeline to at least check whether all the numbers/graphic of an article are reproducible from data provided by authors~\citep{schulz_is_2022}.

This leads me to discuss how researchers can share persistent data with available online archives, such as Zenodo. I myself used Zenodo to share a larger amount of additional data, and provide everything necessary for reproducibility. Zenodo is free and was built and developed ten years ago by researchers to promote Open Science and Open Data as part of the OpenAIRE project. It allows researchers to share data to which a DOI is attributed for each change made to the repository. Thus, one can track the version of the scientific paper, linked to the version of the Zenodo archive. In these repositories can be shared many tools that have been and are developed to enable more reproducible research. Meaning that processed files and results can be reproduced based on the source data using the same program and the same version. 


\subsection{Reproducibility}

Multiple tools can be used and nested together. For example, the pipeline used in a bioinformatic analysis can be described in a snakemake file~\citep{koster_snakemakescalable_2012}. Snakemake is a Python based workflow management program with which a bioinformatician describes the different step, program, codes used to produce each file, resulting in a tree structure of the pipeline. Then, by mentioning what output a researcher expects, the snakemake will identify a chain of jobs to be executed, parallelize them if possible, and re-execute those that are obsolete due to corrupted output files. Snakemake can be used on clusters, composed of high-performance resources, which provide a powerfull means for large scale study. Other similar programs than Snakemake are used for workflow management such as Nextflow~\citep{di_tommaso_nextflow_2017}, or Galaxy~\citep{galaxy_community_galaxy_2022}, but Snakemake seems to be the prominent one recently~\citep{cokelaer_reprohackathons_2023}.

Such workflow management programs can use other informatics tools: compartmentalized micro environment, such as conda, docker or singularity. In conda a user can create an environment, similar to micro virtual machine (VM), that can be shared to others in order to run scripts and program in the same framework. For my usage conda was at some point too slow due too a lot of different environments on my computer resources, so I switches to an alternative solution by using container in which are nested program with the required environment. This container can be loaded from an image spontaneously to run an analysis. Two main programs are used, singularity and docker. Contrary to conda that is dependent of python, images are built at the OS level, which allow more reproducibility power and are easier to share.

Another layer can be used to appreciate all the changes that have been made in a repository or a pipeline. Indeed, eventually the bioinformatician can provide Zenodo archive with different versions, but also it can be accompanied or linked to a Git repository which is a web versioning tool. This means that each change to the codes can be traced back, commented and documented. Git allows the user to tag certain states of the Git repository, which can then be easily collected by Zenodo to be stored in an archive.

All these developed tools represent an excellent opportunity for science to be more reproducible than ever. The limits will still be to define the level of detail of the data to be provided; confidentiality clauses, which may limit sharing; and the time required to restart the analysis. But this is still an opportunity for readers to get crazy details about the data production process and the data relating to the direct figures of the paper. 

Hopefully these tools will be maintained, and journals themselves will provide these services to allow the maintenance of data relating to the articles they publish and for which, in one way or another, they have the responsibility, in order to maintain appropriate ethics.


\section{Perspectives}

It seems clear that some minds are hard to convince, even with the growing numbers of evidences rejecting hypotheses. This has been the case for the debate over alternative splicing products, as noted previously. And it is the case for codons optimizing translation in humans, as articles often discuss its existence but are often misguided and ignore non-adaptive hypotheses. Thus, it is our/my responsibility to approach the questions with rigorous and comprehensive protocols, and to describe my observations as they are, which will potentially yield convincing arguments.

I will delve into new scientific analyses that can be conducted to assure us and the community that our findings are robust and should be considered, as new problematics arise. Also, I will try to offer new avenues for studying variation in genomes architecture and its relationship with random  \gls{genetic drift}.

 
\subsection{Elucidating alternative splicing role}

One of the most debated topic on which I was working on was alternative splicing supposed to be mostly non-functional in humans. To me, the emerging field of third generation sequencing appears to be an opportunity to incorporate long RNA molecules into study, which may ultimately provide access to the full \acrshort{mRNA} molecule~\citep{logsdon_long-read_2020}. Indeed, in our work, because we studied short reads that limit us in the detection of more than one intron \textit{per} read, we made a strong assumption that alternative variants are independent from one intron to another. With such long \acrshort{mRNA} molecules it might be interesting to examine the dependency between variants. Additionally, because what we care about is the \textit{per}-gene alternative splicing rate, accessing the entire \acrshort{mRNA} molecule could allow us to improve the estimation and not making it on the hypothesis that variants are independent. This technique could help us incorporating other alternative splicing events such as intron retention. In my work, the use of short reads (100 bp) has limited me to detect full-length intron retention because in humans, for example, they are larger than 1 \acrshort{kb}. Nonetheless, I attempted to capture intron retention by measuring unspliced reads at splice sites. However, this estimate was strongly influenced by RNA sequencing protocols and noise due to pre-\acrshort{mRNA}. With a complete \acrshort{mRNA} molecule, one can examine intron retention in mature \acrshort{mRNA} (\textit{i.e.} \acrshort{mRNA} exhibiting a poly(A) tail or splicing events).

The limitation of using third generation sequencing has been its large proportion of errors, 1 in every 10 bases (8–15\%), compared to illumina short reads sequencing (1\% error rate)~\citep{morisse_scalable_2021}. A lot of work is being invested to improve the precision of sequencing and correction programs~\citep{luo_vechat_2022}. Recent works obtain a quality score of Q20 (1\% error rate) with some reaching Q30 (0.1\% error rate). However, in our case, to quantify \acrshort{AS} we need to map the reads to the genome, and because the mapping is robust to some sequencing errors, Q20 is already more than sufficient for our purpose. Thus, long-reads sequencing is a more than interesting opportunity.

Another study that might be interesting could be to do a meta-analysis on paper results. Indeed, papers showing no evidence of functional \acrshort{AS} are more likely to not be published than those showing satisfactory results. Perhaps with the help of machine learning and word processing, estimating how many papers show functional \acrshort{AS} variants could be done in a near future. Similarly, a survey of laboratory studies of \acrshort{AS} may be relevant in determining how many studies searching for functional \acrshort{AS} have been inconclusive. These results could provide more perspective and open more dialogue on the subject.


\subsection{Digging in why some species do translational selection}

Although \gls{translational selection} is rare in metazoans, my results did not really capture the biological reason why it varies in large \acrshort{Ne}~species. First, it seems interesting to focus the study on species where we observed translational selection, in order to unravel what determines the variations in \acrshort{TS} intensity in these species. To me, it appears that the Diptera clade is a good candidate, as it is a well studied clade, with numerous species (\textit{i.e.} to date I identified at least 95 species for which genomes are available; \hyperref[suppfig:AppC1]{Appendix C Fig. 1A}), with a wide variation in genomic \gls{GC-content} across species (\hyperref[suppfig:AppC1]{Appendix C Fig. 1D}). Since \acrshort{RNA}-seq samples are not available for all dipterans, I suggest using \textit{Drosophila melanogaster} as a reference species and assigning its gene expression level to corresponding genes in other species. Indeed, the gene expression appears to be conserved between species for homologous genes (\textit{i.e.} reciprocal blast hits; \hyperref[suppfig:AppC1]{Appendix C Fig. 1B,C}). With this in hands, it seems affordable to replicate our previous estimate of the population-scaled selection coefficient (\acrshort{S}). This could reassure us that translational machinery varies depending on the genome base composition (\hyperref[suppfig:AppC1]{Appendix C Fig. 1F}). Furthermore, by using \acrshort{Ne}~proxies, we could observe whether \acrshort{Ne}~is the main drivers of the \acrshort{TS} intensity or not in this clade. 

Because the “drift barrier” hypothesis made prediction for genome characteristics reaching selection/drift equilibrium, it seems interesting to test whether these genomes are indeed at \gls{translational selection} equilibrium. To study if there is an enrichment/diminishment of \acrshort{POC} in a genome, one can study the number of \acrshort{POC} to non-\acrshort{POC} substitutions compared to non-\acrshort{POC} to \acrshort{POC} substitutions.

Finally, if the growth rate of a species is a parameter having an impact on \gls{translational selection}, we could consider capturing its level. This could be done either qualitatively, \textit{e.g.} using hemimetabolous, \textit{i.e.} slow growth rate, versus holometabolous, \textit{i.e.} rapid growth rate. Or quantitavely by estimating the relative growth rate (RGR), which is the rate of growth \textit{per} unit time relative to the size.

Also, if our prediction are correct, and that \acrshort{Ne}~explain some of the variations of \acrshort{TS}, it should be of interest to study species with less intense \gls{genetic drift}, maybe outside of the metazoans range, looking at unicellular eukaryotes for example \citep{lynch_divergence_2023}.


\subsection{Estimating \Ne}

Studying the impact of random  \gls{genetic drift} on genome characteristics is challenging, and the data I used were not a perfect fit for this study, which could undermine the confidence in the findings. For me, one of the most still debated knowledge is the measure of \acrshort{Ne}, \textit{i.e.} the \gls{genetic drift} intensity, which often seems abstract~\citep{waples_what_2022}. This sometimes complicates the interpretation and the messages of the papers studying the effect of \acrshort{Ne}~on genomes evolution. Indeed, there are still many assumptions regarding the measurement of \acrshort{Ne}.

In my study I used four indirect proxies, that are far from perfect. Notably, the three life history traits (LHT) are proxies of the census size, which in small \acrshort{N} populations, such as mammals is expected to be correlated with \acrshort{Ne}. However, if other parameters change (\textit{e.g.} the reproductive mode or the sex ratio), then we don't know is we can predict \acrshort{Ne}~variations based on the life history traits (\textit{i.e.} body mass, body length, longevity). Investigating how changes in these parameters affect \acrshort{Ne} and LHT could clarify how to use this proxies.

Also, the $dN/dS$ is expected to be related to $4$\acrshort{Ne}\acrshort{s}, but it is based on some assumptions. The first one is that \gls{synonymous} codons are neutral but we know that they are not. Thus, we may underestimate the \acrshort{Ne}~with $dN/dS$ proxy in species where \gls{synonymous} codons are selected. Nevertheless, it seems that in metazoans \gls{synonymous} codons are mostly driven by non-adaptive processes due to their strong relation with \gls{GC-content}. However, $dN/dS$ is also used to detect positive selection, thus one assumption is that it is sufficiently rare to posit that \gls{non-synonymous} \gls{substitution}s are mostly deleterious. In GTDrift we discussed about the limits regarding the impact of polymorphism and saturation on this estimator. With the arising amount of genetic data it will be soon available if not already, to have polymorphism in populations ($\acrshort{piS} = 4$\acrshort{Ne}~\acrshort{mu}), which with the specific mutation rate, will give direct measure of short-term \acrshort{Ne}~\citep{lynch_divergence_2023}. At this point it may be relevant to understand how $dN/dS$ fluctuate with \acrshort{piS}, and how the polymorphism affects part of the $dN/dS$ that should be estimated only on substitutions.

Soon the project NeGA should produce data answering the limitation of the imperfect estimate of \acrshort{Ne} to investigate its impact on genomes architecture. Indeed, the project will bring together dozens of pairs of closely related species with a decrease in \gls{effective population size}. Five biological models will be proposed, including Asellidae isopods, passerine birds, Drosophila, swallowtail butterflies and ants. In each pair there has been a shift in the ecological niche which is followed by a decrease in \gls{effective population size}, \textit{e.g.} in Asellidae isopods there are isopods living in surface and ground water. The subterranean species is expected to have a reduced \gls{effective population size}~\citep{lefebure_less_2017}. For each pair it will be interesting to study the evolution of the \acrshort{AS} rate. If our hypothesis is correct, we should expect for each pair an increase in the \acrshort{AS} rate in species with a reduction in \gls{effective population size}. The question remains if the equilibrium between drift and selection for alternative splicing rate/error will be reached in a large time scale or not.

Another parameter to predict \acrshort{Ne}~could involve estimating population density using collaborative databases such as Global Biodiversity Information Facility (GBIF) to collect public observations. I tried to initiate a project aimed at gathering such data, but they appeared to be very heterogeneous and mostly focused on birds. Moreover, even within the bird dataset it seems that this may be biased towards public knowledge of birds. Thanks to machine learning, it might be possible to collect images and automatically annotate them, thereby offering more unbiased data.

Overall, conducting an integrative study utilizing the maximum number of available \acrshort{Ne}~proxies could help determine the reliability of each and the conditions under which we can have confidence in them. This presents an exciting opportunity to predict the absolute \gls{effective population size} (\textit{i.e.} the number of individual in a Hardy-Weinberg population that would yield equivalent patterns of random fluctuations at neutral sites).


\subsection{How \Ne~impact genomes architecture}

During my thesis, I investigated various genomic characteristics. However, there is potential for further exploration in the near future utilizing the resources offered by GTDrift \citep{benitiere_gtdrift_2024}. A comparative analysis could shed light on the impact of \acrshort{Ne}~on variation in size and number of the major isoform introns. I rapidly observed from GTDrift web interface that in hymenopterans the median intron length seems impacted by \acrshort{Ne}~variations (\hyperref[suppfig:AppC2]{Appendix C Fig. 2B}), but on large scale study among vertebrates the intron length variations does not seem to be affected by \acrshort{Ne}~and is conserved within clades (\textit{i.e.} 850 \acrshort{bp} for birds, 300 \acrshort{bp} for fishes and 1,200 \acrshort{bp} for mammals; \hyperref[suppfig:AppC2]{Appendix C Fig. 2A}). Additionally, studying genome size and other dominant genomes architecture parameters would be valuable avenues of investigation. Although it has been shown not to be affected by \acrshort{Ne}~at the metazoan scale \citep{whitney_did_2010, roddy_mammals_2021, marino_effective_2024}, it appears that Hymenoptera genomes size might be explained by \acrshort{Ne}~variations (\hyperref[suppfig:AppC2]{Appendix C Fig. 2D}).

Also, I would like to pursue the investigation of drift impact on genomes architecture in embryophytes, as some species are available in GTDrift data resource.
But I've had trouble getting \acrshort{Ne}~proxies for these species. The ratio $dN/dS$ can provide one but as said before, it's reassuring to have several estimates of \acrshort{Ne}, and for plants, what should be used is not well established. The periods of flowering and survival of plants could be a line of inquiry: \textit{i.e.} annual, multi-annual, biennial plants, \textit{etc.} This will then allows us to re-investigate our previous observations in a new clades. Nevertheless, new problematics could be faced because \acrshort{AS} is very different in plants and metazoans, \textit{e.g.} intron retention is enriched in plants (up to 56\%; \citet{reddy_deciphering_2012, reddy_complexity_2013}) while exon skipping is most prominent in humans (58\%).

\subsection{Environmental cost of research}

Now more than ever, it is imperative to be aware of the environmental impact of our research. During my work in the laboratory, significant computing resources were used, particularly for the alignment of RNA-seq samples on genomes, which could span a week using 16 cores. Additionally, resource-intensive analyses were conducted on the clusters, including phylogenetic tree inference, $dN/dS$ estimation, gene expression profiling, systematic analyses across multiple species \textit{etc.}

In total, my research represented 3,189,232 hours of CPU usage on the computing facilities of CC LBBE/PRABI and the Core Cluster of the Institut Français de Bioinformatique (IFB). By estimating at 1,260 kgs the construction and transport of a 16-core server which has a 7-year lifespan (source: Eco-info \url{https://ferme.yeswiki.net/Empreinte/?PagePrincipale}), and by considering that each core consumes 23.9 W per hour, with a corresponding carbon emission of 79 gCO2/kWh (source: Agence de la transition écologique), the carbon footprint is approximately 3 gCO2/h (1.1 gCO2/h from transport and construction, and 1.9 gCO2/h from electricity). My research was therefore responsible for the emission of 10 tons of CO2, equivalent to 3 Paris - New York round-trip, or 19 Paris - Nice round-trip by plane (based on estimates by \citet{ayoun_quelle_2021}).

This highlights the importance of meticulously building protocols before running lots of unnecessary calculations, optimizing scripts, and checking for errors. Additionally, sharing data allows others to use it without the need for redundant analyses, and publishing negative results helps identify ineffective methodologies. Above all, it seems important to find a balance between robustness of our analyzes and their energy cost. In the field of bioinformatics, given the vast availability of genomic data and the ease of running calculations, the temptation to conduct repetitive analyses on large quantities of data is omnipresent, one click away. Indeed, at some point what is working is not our brain anymore, but, day and night, our computers.

This was one of the main motivations for me to share and publish my data with as much information as possible so that others could replicate the analysis and understand exactly what was done, without having to re-run heavy analyses. In the near future, perhaps research funding agencies will require that a project's carbon emissions be estimated as is done for budgeting.

\subsection{Accessibility}

Lastly, I would like to present in a few words my point of view, which is obviously questionable, about the accessibility not only to the data pertaining to analysis, but the science itself. On this matter, trying to develop a model/protocol that is as simple as possible to address a problem seems to have a better chance of convincing a wide audience. One way to find a compromise on qualitative, ethical science, accessible through both reproducibility and scientific knowledge, could be to focus science more on methods than on results. Meaning defining a primary question; addressing this question with a peer-validated protocol to produce and interpret results. This could avoid extreme observation biases that can be encountered in bioinformatics due to the possibility of changing the protocol and rerunning entire pipeline within a few days in unintentional search for satisfactory results.
These published methods/articles could potentially show negative and positive results, be shorter and therefore more accessible. As I observed during my thesis, it is typically from this perspective that alternative splicing seems to be put forward as being mainly functional. Indeed, people mainly publish positive results which overshadow the negative results that many might observe.

Adopting this `method' approach could have the privilege of scientifically recognizing methods that did not work, not making the same mistakes, ultimately reducing unnecessary carbon emissions, and perhaps being published and revised more quickly to ultimately be more satisfying. 