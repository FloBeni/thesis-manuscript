\thispagestyle{empty}
\vspace*{-70pt}

\begin{center}
	\Large
	\textbf{The impact of random genetic drift on the evolution of genes base composition and expression complexity in metazoans}
\end{center}

\section*{\large{Abstract}}
Genomes exhibit remarkable diversity, varying in size, base composition, expression, number of genes \textit{etc}. Understanding the origin of these changes captivates. Three evolutionary forces have shaped genomes: mutation, \textit{i.e.} the creation of new alleles; selection, \textit{i.e.} the impact of alleles on fitness; random genetic drift, \textit{i.e.} the random sampling of alleles through generations. While, over time, beneficial mutations (\textit{i.e.} which have contributed to the adaptation of species to their environment) have become fixed in the genomes, natural selection alone cannot explain all the diversity observed at the molecular level.

Instead, the neutral theory of evolution posits that mutations with negligible effects on individual fitness also play a significant role in shaping genome evolution. Specifically, the nearly neutral theory suggests that the efficiency to purge (or fix) slightly deleterious (or advantageous) mutations, depends on the ability of selection to overcome drift hazard. If selective pressure on specific biological traits remains constant, a genome's ability to attain optimality becomes limited by drift. This hypothesis known as the “drift barrier” predicts that species with strong drift, and thus small effective population size (\Ne), have a poorly optimized genome compared to those with larger effective population sizes.

Throughout my thesis I studied the impact of random genetic drift on genomes architecture in animals. To do so, I collected numerous genomes, from 1,506 species, and transcriptomes, from 15,935 RNA-seq samples, to analyze their transcriptomic diversity and coding sequences composition. These analyses are shared in a bio-informatic data resource along with effective population size proxies. 

One key aspect of my research involved investigating how increasing drift intensity affects transcriptomic diversity, through the study of splicing variants. Our results demonstrated that drift limits the capacity of selection to optimize the splicing machinery in genomes. It ultimately leads to the production of many spurious transcripts in species with small \Ne, such as human.

I investigated another characteristic of genomes base composition, the use of synonymous codons in metazoans. Our research revealed that in species where translational selection (\textit{i.e.} promoting the use of codons optimizing translation process) is detected, the population-scaled selection coefficient is small. This suggests that translational selection can be efficient only in species with large effective population size, elucidating its rarity in metazoans. Nevertheless, intriguingly, certain species with large \Ne~did not show translational selection signals, which implies that the selective advantage in optimizing the translation process varies across species.

Overall, this thesis underscores the impact of drift on genome architecture and provides an interesting conceptual framework, along with a collection of reusable data, to examine what is or is not under selection in our genomes.